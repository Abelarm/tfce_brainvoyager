\documentclass{beamer}

\usepackage[italian]{babel}
\usepackage[T1]{fontenc}
\usepackage{beamerthemebjeldbak}
\usepackage{graphicx}
\usepackage{listings}
\usepackage[utf8]{inputenc} 
\usepackage{epsfig}  
\usepackage{amsmath} 
\usepackage{multicol}
\usepackage{amsfonts}
\usepackage{hyperref}
\usepackage{listings}

\lstset{
  basicstyle=\footnotesize,
  language=C,                % choose the language of the code
  numbers=left,                   % where to put the line-numbers
  stepnumber=1,                   % the step between two line-numbers.        
  numbersep=5pt,                  % how far the line-numbers are from the code
  backgroundcolor=\color{white},  % choose the background color. You must add \usepackage{color}
  showspaces=false,               % show spaces adding particular underscores
  showstringspaces=false,         % underline spaces within strings
  showtabs=true,                 % show tabs within strings adding particular underscores
  tabsize=2,                      % sets default tabsize to 2 spaces
  captionpos=b,                   % sets the caption-position to bottom
  breaklines=true,                % sets automatic line breaking
  breakatwhitespace=true,         % sets if automatic breaks should only happen at whitespace
  title=\lstname,                 % show the filename of files included with \lstinputlisting;
}



\setbeamertemplate{itemize/enumerate body begin}{\footnotesize}

\title{Threshold-Free Cluster Enhancement}
\author{Luigi Giugliano$^1$, Marco Mecchia$^1$}
\institute{$^1$Università degli studi di Salerno}


\begin{document}

\begin{frame}
   \maketitle
\end{frame}

\begin{frame}
  \frametitle{Overview}
  \footnotesize \tableofcontents
\end{frame}

\AtBeginSection[]
  {
     \begin{frame}<beamer>
     \frametitle{Overview}
   \footnotesize \tableofcontents[currentsection]
     \end{frame}
}


\section{Codice}
\begin{frame}
\frametitle{Codice}
Andremo ora a spiegare il codice prodotto  per il plugin TFCE.\\
\medskip
I file principali che compongono il plugin sono:
\begin{itemize}
\item{Tfce.cpp}
\item{Utilities.cpp}
\end{itemize}
\textbf{Tfce} è il core del plugin, dove avviene il  calcolo degli score.\\
\smallskip
\textbf{Utilities} invece contiene tutti le funzioni di supporto per l'esecuzione del plugin stesso.
\end{frame}

\begin{frame}[fragile]
L' unica funzione che viene esposta dal file \textbf{Tfce.h} è:
\begin{center}
\begin{lstlisting}
#ifndef TFCE_H
    #define TFCE_H
    #include <float.h>
    float * tfce_score(float * map, int dim_x, int dim_y, int dim_z, float E, float H, float dh);
#endif //TFCE_H
\end{lstlisting}
\end{center}
\end{frame}

\begin{frame}[fragile]
Le funzioni che espone \textbf{Utilities.h} sono:
\begin{center}
\begin{lstlisting}
#ifndef UTILITIES_H
    #define UTILITIES_H

    #include <float.h>
    #include <stdio.h>

    void findMinMax(float *map, int n, float *min, float *max, float * range);

    int confront(float a, float b, char operation);

    int * getBinaryVector(float * map, int n, int (*confront)(float, float), float value, int *      numOfElementsMatching);
\end{lstlisting}
\end{center}
\end{frame}

\begin{frame}[fragile]
\begin{center}
\begin{lstlisting}
   float * fromBinaryToRealVector(float * map, int n, int * binaryVector);

   float * fill0(int n);

   void apply_function(float * vector, int n, float (* operation) (float a, float b), float argument);

   int linearIndexFromCoordinate(int x, int y, int z, int max_x, int max_y);

   void coordinatesFromLinearIndex(int index, int max_x, int max_y, int * x, int * y, int * z);

   float * copyAndConvertIntVector(int * vector, int n);
   
#endif //UTILITIES_H
\end{lstlisting}
\end{center}
\end{frame}

\begin{frame}[fragile]
\textbf{Tfce.cpp} oltre al metodo visto precedentemente fornisce i seguenti metodi:\\
\begin{lstlisting}
int * find_clusters_3D(int * binaryVector, int dim_x, int dim_y, int dim_z, int n, int * num_clusters)
\end{lstlisting}
questo metodo preso in input una \textbf{mappa binaria in 3D ma linearizzata}, le sue tre dimensioni, il numero totale voxel della mappa e il puntatore ad un intero che indica il numero attuale di cluster trovati.\\
\smallskip
Cercando i cluster all'interno della mappa 3D utilizzando la \textbf{26-connectivity}
\end{frame}

\begin{frame}
Restituisce un'altra mappa in cui al posto degli uno viene sostituito l'identificativo del cluster.
\[
\begin{bmatrix}
    1       & 1 & 0 & 1 & 1\\
    1       & 1 & 0 & 0 & 0 \\
    1       & 0 & 1 & 1 & 1 \\
    1       & 0 & 0 & 1 & 1
\end{bmatrix}
=
\begin{bmatrix}
    1       & 1 & 0 & 3 & 3\\
    1       & 1 & 0 & 0 & 0 \\
    1       & 0 & 2 & 2 & 2 \\
    1       & 0 & 0 & 2 & 2
\end{bmatrix}
\]
In questo piccolo esempio in 2D viene mostrato il funzionamento della nostra funzione.
\end{frame}

\end{document}
